\documentclass{article}

\usepackage{hevea}
\usepackage{tabularx}


%
% commands
%

\ifhevea

% web only
\renewcommand{\url}     [1]   {\ahrefurl{#1}}
\newcommand{\link}      [2]   {\ahref{#1}{#2}}
\newcommand{\sel}       [1]   {\textcolor{red}{\code{\textbf{#1}}}}
\newcommand{\slabel}    [2]   {\label{sec:#1}#2}
\newcommand{\tocitem}   [2]   {\item\ahrefloc{sec:#1}{#2}}
\newcommand{\index}     [1]   {}  % hevea doesn't know \index

\else

% latex only
\let\realurl=\url
\renewcommand{\url}     [1]   {{\small\realurl{#1}}}
\newcommand{\link}      [2]   {#2}

\fi

\renewcommand{\code}         [1] {{\upshape\texttt{#1}}}
\newcommand{\selectspecurl}      {http://www.ircam.fr/anasyn/sdif/extern/utilities-doc.html#selectspec} 
\newcommand{\selectspeclink} [1] {\link{\selectspecurl}{#1}} 
\newcommand{\commentp}       [1] {%
	\begin{rawhtml}<table><tr valign=top><td>/* <td>\end{rawhtml}
 	\textit{#1}~*/
	\begin{rawhtml}</table>\end{rawhtml}}
%\newcommand{\commentp}       [1] {\begin{tabular}{lp{}}
%				  /* & \textit{#1}~*/\\
%				  \end{tabular}}
\newcommand{\commentl}       [1] {/* \textit{#1} */}
\renewcommand{\@bodyargs}{TEXT=black BGCOLOR=white}

%
% title and abstract
%

\title  {IRCAM SDIF Library Tutorial}
\author {\link{mailto:schwarz@ircam.fr}{Diemo Schwarz}}
\date   {\today}


\begin{document}
\maketitle



\section {Introduction}

A little note on the \key{naming convention} of SDIF funtions: All functions
starting with \code{SdifF}... work on an open SDIF file.  They all take an
\code{SdifFileT *} as first argument.

The library is initialised by calling

\begin{alltt}
SdifGetInit ("");
\end{alltt}

giving an optional \code{.styp} file with description of type definitions.  An
empty string \code{""} means take the file at the standard place, or use the
file given by the environment variable \code{SDIFTYPES}, if it exists.


\section {Reading}

%\subsection {Making Your Own Loop}

\begin{alltt}

\commentp{Open file for reading and parse selection.  (See
	  \selectspeclink{Utilities} for selection syntax.)}
SdifFileT *file = SdifFOpen ("filename::\selectspeclink{\var{selection}}", eRead);
SdifFReadGeneralHeader  (file); \commentl{Read header}
SdifFReadAllASCIIChunks (file); \commentl{Read header info}
if (SdifFLastError (file))	\commentl{Check for errors}
\{ 
    exit (1);
\}

\commentp{Read all frames matching the file selection.}  
while (SdifFReadNextSelectedFrameHeader (file))
\{
	SdifFloat8	time 	 = SdifFCurrTime (file);
	SdifSignature	sig  	 = SdifFCurrFrameSignature (file);
	SdifUInt4	streamid = SdifFCurrID (file);
	SdifUInt4	nmatrix  = SdifFCurrNbMatrix (file);

	/* Read all matrices in this frame matching the selection.
           Specific rows and columns can be selected.  If it is not
           NULL, the matrix selection msel is added to the file
           selection, given with SdifFOpen, and can be differnt for
           each call. */
SdifFCurrMatrixIsSelected
	while (SdifFReadNextMatrix (file, msel))
	\{
 		SdifSignature	sig   = SdifFCurrMatrixSignature (file);
		SdifInt4	nrows = SdifFCurrNbRow (file);
		SdifInt4	ncols = SdifFCurrNbCol (file);

		SdifDataTypeET	type  = SdifFCurrDataType (file);
		void*		data  = SdifFCurrMatrixData (file);

		/* do something with data pointer directly (data is
                   stored in row-major order), or access values by
                   indices (1 <= row <= nrows, 1 <= col <= ncols): */
		SdifFloat8 	value = SdifFCurrDataValue (file, row, col);
	\}
\}

if (SdifFCheckError (file))   
{ /* complain about error */
  char errmsg [SdifMaxErrMsg];
  SdifFGetErrorMsg (file, errmsg);
  /* print it and exit or return */
}

SdifFClose (file);

SdifGenKill ();
\end{alltt}


\subsection {Let the Lib Loop}

SdifFReadFileSimple ("filename::selection", MatrixFunction);

SdifFReadFileSimple ("filename::selection", MatrixFunction);

\section {Writing}


\end{document}
% --> cocoon anchor
