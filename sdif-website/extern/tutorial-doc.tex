\documentclass{article}

\usepackage{hevea}
\usepackage{tabularx}
\usepackage{color}


%
% commands
%

\ifhevea

% web only
\renewcommand{\url}     [1]   {\ahrefurl{#1}}
\newcommand{\link}      [2]   {\ahref{#1}{#2}}
\newcommand{\sel}       [1]   {\textcolor{red}{\code{\textbf{#1}}}}
\newcommand{\Section}   [1]   {\\\\\\\label{#1}\section#1}
\newcommand{\tocitem}   [2]   {\item\ahrefloc{sec:#1}{#2}}
\newcommand{\index}     [1]   {}  % hevea doesn't know \index

\else

% latex only
\let\realurl=\url
\renewcommand{\url}     [1]   {{\small\realurl{#1}}}
\newcommand{\link}      [2]   {#2}

\fi

\renewcommand{\code}         [1] {{\upshape\texttt{#1}}}
\newcommand{\selectspecurl}      {http://www.ircam.fr/anasyn/sdif/extern/utilities-doc.html#selectspec} 
\newcommand{\selectspeclink} [1] {\link{\selectspecurl}{#1}} 
\newcommand{\commentf}       [1] {\textcolor{blue}{\texttt{\textit{#1}}}}
\newcommand{\commentp}       [1] {%
	\@print{<table cellspacing=0 cellpadding=0><tr valign=top><td>}%
	\commentf{/* }\@print{<td>}\commentf{#1~*/}\@print{</table>}}
%\newcommand{\commentp}       [1] {\begin{tabular}{lp{}}
%				  /* & \textit{#1}~*/\\
%				  \end{tabular}}
\newcommand{\commentl}       [1] {\commentf{/* #1 */}}
\renewcommand{\@bodyargs}{TEXT=black BGCOLOR=white}


%
% title
%

\title  {IRCAM SDIF Library Tutorial}
\author {\link{mailto:schwarz@ircam.fr}{Diemo Schwarz}}
\date   {\today}


\begin{document}
\maketitle



\Section {Introduction}

This tutorial describes the standard way to read and write an SDIF file using
the \link{http://www.ircam.fr/sdif}{Ircam SDIF Library}.  Only rudimentary
error checking is done.  However, the reading program uses the SDIF selection
(see this \link{http://www.ircam.fr/anasyn/schwarz/publications/icmc2000/#sec:selection}{ICMC article}) to show you how easy it is to incorporate powerful selection capabilities into your programs.

To use the sdif library, include the definitions with

\begin{alltt}
#include <sdif.h>
\end{alltt}

and link with the SDIF library using \code{-lsdif}.

SDIF signatures are the 4 characters that identify a frame or matrix type.  To
create and read them, use:

\begin{alltt}
SdifSignature mysig  = SdifSignatureConst ('1FQ0');
char*         sigstr = SdifSignatureToString (mysig);
\end{alltt}

A little note on the \key{naming convention} of SDIF funtions: All functions
starting with \code{SdifF}... work on an open SDIF file.  They all take an
\code{SdifFileT *} as first argument.





\Section {Initialisation}

The library is initialised by calling

\begin{alltt}
SdifGenInit ("");
\end{alltt}

giving an optional \code{.styp} file with description of type definitions.  An
empty string \code{""} means take the file at the standard place, or use the
file given by the environment variable \code{SDIFTYPES}, if it exists.

To deinitialise and clean up, call:

\begin{alltt}
SdifGenKill ();
\end{alltt}





\Section {Reading}

%\subsection {Making Your Own Loop}

\begin{alltt}

size_t  bytesread = 0;
int	eof       = 0;	\commentl{End-of-file flag}

\commentp{Open file for reading and parse selection.  (See
	  \selectspeclink{Utilities} for selection syntax.)}
SdifFileT *file = SdifFOpen ("filename.sdif::\selectspeclink{\var{selection}}", eReadFile);
SdifFReadGeneralHeader  (file); \commentl{Read file header}
SdifFReadAllASCIIChunks (file); \commentl{Read ASCII header info, such as name-value tables}

\commentp{Read all frames matching the file selection.}  
while (!eof  &&  SdifFLastError (file) == NULL)
\{
    bytesread += SdifFReadFrameHeader(file);

    \commentl{search for a frame we're interested in}
    while (!SdifFCurrFrameIsSelected (file)
	   ||  SdifFCurrSignature (file) != mysig)
    \{
        SdifFSkipFrameData (file);
        if ((eof = SdifFGetSignature(file, &bytesread) == eEof))
            break;                                         \commentl{eof}
        bytesread += SdifFReadFrameHeader(file);
    \}

    if (!eof)
    \{    \commentl{Access frame header information}
    	SdifFloat8      time     = SdifFCurrTime (file);
    	SdifSignature   sig      = SdifFCurrFrameSignature (file);
    	SdifUInt4       streamid = SdifFCurrID (file);
    	SdifUInt4       nmatrix  = SdifFCurrNbMatrix (file);
    	int		m;
    
    	\commentl{Read all matrices in this frame matching the selection.}
    	for (m = 0; m < nmatrix; m++)
    	\{
    	    bytesread += SdifFReadMatrixHeader (file);
    
    	    if (SdifFCurrMatrixIsSelected (file))
    	    \{    \commentl{Access matrix header information}
 		SdifSignature	sig   = SdifFCurrMatrixSignature (file);
		SdifInt4	nrows = SdifFCurrNbRow (file);
		SdifInt4	ncols = SdifFCurrNbCol (file);
		SdifDataTypeET	type  = SdifFCurrDataType (file);
    
		int		row, col;
		SdifFloat8	value;
    		
    		for (row = 0; row < nrows; row++)
    		\{
		    bytesread += SdifFReadOneRow (file);
    
    		    for (col = 1; col <= ncols; col++)
    		    \{
			\commentl{Access value by value...}
    			value = SdifFCurrOneRowCol (file, col);
    
			\commentl{ Do something with the data...}
		    \}
    		\}
    	    \}
    	    else
	    \{
    		bytesread += SdifFSkipMatrixData(file);
	    \}
    	    
    	    bytesread += SdifFReadPadding(file, SdifFPaddingCalculate(file->Stream, bytesread));
    	\} 
    
        \commentl{read next signature}
    	eof = SdifFGetSignature(file, &bytesread) == eEof;
    \}
\}

if (SdifFLastError (file))   \commentl{Check for errors}
\{
    exit (1);
\}

SdifFClose (file);
\end{alltt}




\Section {Writing}

\begin{alltt}
SdifFileT *file = SdifFOpen("filename.sdif", eWriteFile);  \commentl{Open file for writing}
\commentl{Here, you could add some text data to the name-value tables}
SdifFWriteGeneralHeader  (file);    \commentl{Write file header information}
SdifFWriteAllASCIIChunks (file);    \commentl{Write ASCII header information}

\commentp{Writing can be done in different flavours:  For the common case of a frame with one matrix, use:}
SdifFWriteFrameAndOneMatrix (file, mysig, mystream, mytime,             \commentl{frame header}
				   mysig, eFloat4, nrows, ncols, data); \commentl{matrix}

\commentp{If you have one or more matrices ready in row-major order, use:}
SdifUInt4 framesize = SdifSizeOfFrameHeader () 
                    + SdifSizeOfMatrix (eFloat4, nrows, ncols); \commentl{as many as you like...}
SdifFSetCurrFrameHeader (file, mysig, framesize, 1, mystream, mytime);
SdifFWriteFrameHeader (file);
SdifFWriteMatrix (file, mysig, eFloat4, nrows, ncols, data);

\commentp{If you want to calculate your data on-the-fly, you can write it the nitty-gritty way:}
SdifUInt4 SizeFrameW = 0;

SdifFSetCurrFrameHeader (file, mysig, _SdifUnknownSize, nmatrix, mystream, mytime);
SizeFrameW += SdifFWriteFrameHeader (file);

\commentl{Write matrix header}
SdifFSetCurrMatrixHeader (file, mysig, eFloat4, nrows, ncols);
SizeFrameW += SdifFWriteMatrixHeader (file);

\commentl{Write matrix data}
for (row = 0; row < nrows; row++)
\{
    for (col = 1; col <= ncols; col++)
    \{
        SdifFSetCurrOneRowCol (file, col, value);
        SizeFrameW += SdifFWriteOneRow (file);
    \}
\}
\commentl{Write matrix padding}
SizeFrameW += SdifFWritePadding (file, SdifFPaddingCalculate (file->Stream, SizeFrameW));

SdifUpdateChunkSize (file, SizeFrameW - 8);  \commentl{Set frame size}

SdifFClose (file);
\end{alltt}




\Section{Goodies}

Some functions to make life with SDIF even easier:

\begin{alltt}
\commentp{Check if the file exists and is a good SDIF file}
if (SdifCheckFileFormat ("filename.sdif"))
    /* ok */;

\commentp{Check if file is good, exists, and contains certain frame types}
SdifSignature wanted [] = \{ SdifSignatureConst('1TRC'), 
			    SdifSignatureConst('1HRM'), 
			    eEmptySignature \commentl{end marker} 
			  \},
	      firstfound;
int           firstindex;

\commentl{This returns the signature of the first wanted frame found}
firstfound = SdifCheckFileFramesTab ("filename.sdif", wanted);

printf ("First sig found is: %s\backslash{n}", 
	firstfound  ?  SdifSignatureToString(firstfound)  
		    :  "none of interest");

\commentl{Alternatively, you can get the index of the first frame found}
firstindex = SdifCheckFileFramesIndex ("filename.sdif", wanted);

printf ("First sig found is: %s\backslash{n}", 
	firstindex >= 0  ?  SdifSignatureToString(wanted [firstindex])  
		         :  "none of interest");
\end{alltt}

N.B.:  All of the above functions require a real file, and do not work with standard input.

\end{document}
% --> cocoon anchor




